Se define la función $\text{mex} : \mathcal{P}(\mathbb{N}) \rightarrow \mathbb{N}$ como
\[\text{mex}(X) = \min\{j : j \in \mathbb{N} \land j \notin X\}\]
Intuitivamente, $\text{mex}$ devuelve, dado un conjunto $X$, el menor número natural que no está en $X$.
Por ejemplo, $\text{mex}(\{0, 1, 2\}) = 3$, $\text{mex}(\{0, 1, 3\}) = 2$ y $\text{mex}(\{1, 2, 3, \ldots\}) = 0$.

Dado un vector de números $a_1, \ldots, a_n$, queremos encontrar la permutación $b_1, \ldots, b_n$ de los mismos que maximice
\[\sum_{i=1}^{n} \text{mex}(\{b_1, \ldots, b_i\})\]

Por ejemplo, si el vector es $\{3, 0, 1\}$ podemos ver que la mejor permutación es $\{0, 1, 3\}$, que alcanza un valor de
\[\text{mex}(\{0\}) + \text{mex}(\{0, 1\}) + \text{mex}(\{0, 1, 3\}) = 1 + 2 + 2 = 5\]
